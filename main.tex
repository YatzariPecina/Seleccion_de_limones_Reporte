% Actualizacion 2024: Logotipos de la UPV

\documentclass[12pt]{article}
\usepackage[left=2.5cm,top=2.0cm,right=2.5cm,bottom=3.0cm]{geometry}
\usepackage[utf8]{inputenc}
\usepackage[spanish]{babel}
%\usepackage[english, american]{babel}
%\usepackage[spanish,es-tabla]{babel}
\usepackage[linguistics]{forest}
\usepackage{amssymb, amsmath, amsbsy} % simbolitos
\usepackage{longtable} % para tablas largas
\usepackage{graphicx}
\usepackage{fancyhdr}
\usepackage{xcolor}
\usepackage{multirow}
\usepackage{listings}
\usepackage{caption}
\usepackage{subcaption}
%\usepackage{parskip}
\usepackage[skip=12pt plus1pt]{parskip}
\usepackage{pdfpages} % Incluir PDF en documento en LATEX
\usepackage{verbatim} % comentarios
\usepackage{algpseudocode}
\usepackage{algorithm}
\usepackage{pdflscape}
\usepackage{multirow}
\usepackage{afterpage}
\usepackage{array,booktabs,ragged2e}
%\newcolumntype{R}[1]{>{\RaggedLeft\arraybackslash}p{#1}}
\newcolumntype{L}[1]{>{\raggedright\let\newline\\\arraybackslash\hspace{0pt}}m{#1}}
\newcolumntype{C}[1]{>{\centering\let\newline\\\arraybackslash\hspace{0pt}}m{#1}}
\newcolumntype{R}[1]{>{\raggedleft\let\newline\\\arraybackslash\hspace{0pt}}m{#1}}

\floatname{algorithm}{Algoritmo}
\renewcommand{\listalgorithmname}{Lista de algoritmos}
\renewcommand{\algorithmicrequire}{\textbf{Entrada:}}
\renewcommand{\algorithmicensure}{\textbf{Salida:}}

% Comentario con respecto a las referencias:
% Por defecto este documento utiliza el formato IEEEtr para formatear las referencias. Si algun asesor requiere formato APA, solo comente la siguiente linea y descomente la linea debajo. Tambien para que las citas funcionen de manera adecuada, el paquete label debe tener las opciones mostradas, o en su defecto, el año en el listado no lo muestra correctamente \usepackage[english, american]{babel}

% Para utilizar el formato de citas IEEE y comentar los dos parrafos siguientes
\usepackage[backend=bibtex,sorting=none]{biblatex}
% Para utilizar el formato APA, sugiero comentar la linea anterior y descomentar las dos proximas lineas
%\usepackage[backend=biber,style=apa]{biblatex}
%\DeclareLanguageMapping{english}{american-apa}

\makeatletter
\DefineBibliographyExtras{spanish}{%
  \setcounter{smartand}{1}%
  \let\lbx@finalnamedelim=\lbx@es@smartand
  \let\lbx@finallistdelim=\lbx@es@smartand
}
\renewbibmacro*{name:delim:apa:family-given}[1]{%
  \ifnumgreater{\value{listcount}}{\value{liststart}}
    {\ifboolexpr{
       test {\ifnumless{\value{listcount}}{\value{liststop}}}
       or
       test \ifmorenames
     }
       {\printdelim{multinamedelim}}
       {\lbx@finalnamedelim{#1}}}
    {}}
\makeatother


% Estas lineas permiten romper los hipervinculos muy largos en las referencias!!!!
\setcounter{biburllcpenalty}{7000}
\setcounter{biburlucpenalty}{8000}
\addbibresource{x.bib} % ARCHIVO DE BIBLIOGRAFÍA


%\usepackage{url}
\usepackage[bookmarks=true,breaklinks=true,bookmarksopen=false,colorlinks=true,linkcolor=blue]{hyperref}
\usepackage[hyphenbreaks]{breakurl}
% Regla que define explicitamente que caracteres rompen los hipervinculos para separar las lineas
%https://es.overleaf.com/11089898rhgykrqyqytx
% Actualiza en automático la fecha de las citas de internet a la fecha de la compilación del documento
\usepackage{datetime}
\newdateformat{specialdate}{\twodigit{\THEDAY}-\twodigit{\THEMONTH}-\THEYEAR}
%\newdateformat{specialdate}{\twodigit{\THEDAY}-\THEYEAR}
\date{\specialdate\today}

\newcommand{\HRule}{\rule{\linewidth}{0.25mm}}


% CONSTANTES NECESARIAS PARA EL DOCUMENTO ---> MODIFIQUEN A SU CRITERIO
\newcommand{\ncarrera}            {Ingeniería en Tecnologías de la Información}
\newcommand{\maestro}{Dr. Said Polanco Martagón}
\newcommand{\NombreAlumnoJesus}{Jesus Antonio Olazarán Mora}
\newcommand{\NombreAlumnoDamaris}{Damaris Alexia Espinosa Castro}
\newcommand{\NombreAlumnoLuisana}{Luisana Guadalupe Rodríguez  Salas}
\newcommand{\NombreAlumnoHer}{Hermayonick Catalina Hernández González}
\newcommand{\NombreAlumnoJose}{José Guadalupe Martínez Herrera}
\newcommand{\NombreAlumnoYatzari}{Yatzari Eduve Pecina Vidales}
%Hombres cambien LA por EL
\newcommand{\elolaNombreAlumno}{el}  
\newcommand{\OA}           {o}  %Hombres cambien A por O
\newcommand{\Matricula}           {1730505}  %SU MATRICULA
\newcommand{\NombreProyecto}{Máquina de selección de limones}
\newcommand{\fechacarta}{26 de Abril de 2021}
\newcommand{\ncuatrimestre}{Septiembre-Diciembre 2024}
\newcommand{\nevalador}{Dr. Hiram Herrera Rivas}
\newcommand{\FechaExposicion}{11 de Agosto de 2021}
\newcommand{\HoraExposionFormatoVenticuatroHoras}{10:00}
\newcommand{\elolaNombreEmpresa}{la}  %Si la empresa es femenino (por ejemplo universidad, usen la) o masculino (el instituto) pongan el
\newcommand{\organismoreceptor}   {Universidad Politécnica de Victoria}

% NOTA: Dos diagonales juntas (\\) indican un saldo de linea. En este caso particular hay 2 (el titulo se ajusta a tres lineas, porque es muy largo. Hacer las adecuaciones pertinentes
\newcommand{\NombreProyectoheader}     {Maquina de selección de limones}
\newcommand{\fechaPortada}               {Octuble del 2024}





\newcommand{\separacionCorta}{0.0cm}
\newcommand{\separacionLarga}{1.2cm}
\newcommand{\separacionParrafo}{0.8cm}

\usepackage[overload]{textcase}
\newcommand{\iemph}[1]{\MakeTextUppercase{#1}}

\pagestyle{fancy}
\headheight 45pt
\fancyhead{} % Clear all header fields
\fancyhead[L]{\includegraphics[height=1.00cm]{UTYP.png}}%
\fancyhead[C]{\begin{center}\NombreProyectoheader\end{center}}%
%\fancyhead[R]{\includegraphics[height=1.5cm]{LogoUPV_2019.png}}%
\fancyhead[R]{\includegraphics[height=1.25cm]{LogoUPV_2023.png}}%
\fancyfoot[R]{\thepage} % Clear all footer fields 
\fancyfoot[C]{}
\fancyfoot[L]{}

\DefineBibliographyStrings{english}{%
  references = {Referencias},% replace "references" with "bibliography"  for `book`/`report`
}

\addto\captionsenglish{%
  \renewcommand{\figurename}{Figura}%
  \renewcommand{\tablename}{Tabla}%
} 

\usepackage{wallpaper}
 
 
%\renewcommand{\figurename}{Figura}
%\renewcommand{\tablename}{Tabla}

 
\begin{document}

%-----------------------------------------------------------------------------------------------------------------
% PAGINA 1 - PORTADA
\setcounter{page}{1}
\pagenumbering{roman}
\thispagestyle{empty}

\begin{center}

\begin{tabular}{cp{5cm}c}
\includegraphics[height=2.25cm]{UTYP.png} & 
& \includegraphics[height=2.25cm]{LogoUPV_2023.png}   \\
\end{tabular}

\Large \textbf{UNIVERSIDAD POLITÉCNICA DE VICTORIA}
\vspace{0.5cm}
\hrule
\vspace{0.1cm} 
\hrule
\vspace{0.5cm}


%\HRule \\[\separacionCorta]
\textbf{\iemph{\NombreProyecto}} \\[\separacionLarga]
%\Large \textbf{TESINA}
%\HRule \\[\separacionLarga]
T E S I N A \\[\separacionLarga]

PRESENTA: \\[\separacionCorta]
%\textbf{\Capitalize{\NombreAlumno}\\[\separacionLarga]
\textbf{\iemph{\NombreAlumnoDamaris}}\\[\separacionCorta]
\textbf{\iemph{\NombreAlumnoHer}}\\[\separacionCorta]
\textbf{\iemph{\NombreAlumnoJose}}\\[\separacionCorta]
\textbf{\iemph{\NombreAlumnoJesus}}\\[\separacionCorta]
\textbf{\iemph{\NombreAlumnoYatzari}}\\[\separacionCorta]
\textbf{\iemph{\NombreAlumnoLuisana}}\\[\separacionLarga]
%EN CUMPLIMIENTO DE \\[\separacionCorta]
%LA ESTADÍA DE LA CARRERA DE \\[\separacionCorta]


MAESTRO \\[\separacionCorta]
\textbf{\iemph{\maestro}} \\[\separacionLarga]

ORGANISMO RECEPTOR \\[\separacionCorta]
\textbf{\iemph{\organismoreceptor}} \\[\separacionLarga]

\end{center}
\begin{flushright}
\iemph{Ciudad Victoria, Tamaulipas, \fechaPortada}
\end{flushright}

\HRule 

%-----------------------------------------------------------------------------------------------------------------
% PAGINA 2 - INDICE

%\clearpage
%\addcontentsline{toc}{section}{Índice}
%\renewcommand\contentsname{Índice}
%\tableofcontents

%-----------------------------------------------------------------------------------------------------------------
% CAPITULOS

\clearpage
\pagenumbering{arabic}
\setcounter{page}{1}

\clearpage

\section{Introducción}
La industria agrícola tiene un proceso delicado que empieza desde la selección hasta la cosecha y almacenamiento. En particular, los limones forman parte de la dieta de muchas personas y sus características logran darle un toque refrescante a comidas y bebidas. Tomar en cuenta la vida útil del limón y la necesidad de ofrecer productos de calidad permite garantizar la reducción del desperdicio y la satisfacción de los consumidores. \\\\
El problema radica en la dificultad de detectar cuándo un limón ha alcanzado el punto óptimo de maduración; en ocasiones este proceso se realiza manualmente y puede generar imprecisión en el proceso además de no ser lo suficientemente rápido. \\\\
En la actualidad, la automatización se ha convertido en una solución para optimizar numerosos procesos, reduciendo así la participación humana y los errores asociados. Una de las áreas que se ha visto involucrada en estos cambios tecnológicos ha sido la agricultura, en la que se ha hecho uso de drones para monitorear el crecimiento de los cultivos, detectar problemas en el campo o realizar tareas programadas; robots para realizar tareas como la cosecha de frutas y verduras o trabajar en terrenos difíciles o peligrosos; sensores y sistemas de control que detecten la humedad del suelo, temperatura y demás factores ambientales para ayudar a los agricultores a ajustar la cantidad de agua y nutrientes que se aplican a los cultivos. \\\\
Con este nivel de tecnología, es que se pueden desarrollar soluciones adecuadas utilizando dispositivos para el procesamiento de imágenes. Entre ellos tenemos la idea de crear una banda transportadora automatizada, en la que se vayan colocando los limones, y una cámara ESP32-CAM, que permita la visualización para la revisión de los limones. Este mecanismo requiere la combinación del hardware y software para reconocimiento de colores que indiquen la madurez del limón, tomando en cuenta nuestro principal objetivo que es mejorar el proceso en la selección de limones disminuyendo el error humano. 


\clearpage
\section{Antecedentes teóricos}

Con el paso de los años, se han visto innovaciones en el campo agrícola, con la llegada del machine learning, se han cambiado los procesos de producción mediante la implementación de sistemas inteligentes. En el caso de la selección de limones se han implementado diferentes modelos por visión computacional para la selección de los limones o mediante el entrenamiento de inteligencias artificiales con fotografías de limones los cuales expertos consideran como de buena madurez y calidad para su distribución.

Para el entendimiento de esta sección y del proyecto en general, es de importancia definir algunos conceptos ya que son la base para el desarrollo e implementación de los sistemas inteligentes.

\textbf{Machine learning (ML):} Es una rama de la inteligencia artificial (IA) y la informática que se centra en el uso de datos y algoritmos para permitir que la IA imite la forma en que los humanos aprenden, mejorando gradualmente su precisión \cite{MachineLearning}

\textbf{Dataset:} Un dataset es una colección organizada de datos que puede incluir números, texto, imágenes o videos, estructurados en filas y columnas. Estos datos se utilizan para análisis y toma de decisiones en diversas disciplinas. Por ejemplo, un dataset de ventas puede contener columnas como Fecha de Venta, Producto, Cantidad, y Precio \cite{Datasets}

\textbf{Limón:} Un dataset es una colección organizada de datos que puede incluir números, texto, imágenes o videos, estructurados en filas y columnas. Estos datos se utilizan para análisis y toma de decisiones en diversas disciplinas. Por ejemplo, un dataset de ventas puede contener columnas como Fecha de Venta, Producto, Cantidad, y Precio \cite{DOF}

\textbf{Norma Oficial Mexicana (NOM):} Las Normas Oficiales Mexicanas (NOM) son regulaciones técnicas de observancia obligatoria expedidas por las dependencias competentes, que tienen como finalidad establecer las características que deben reunir los procesos o servicios cuando estos puedan constituir un riesgo para la seguridad de las personas o dañar la salud humana; así como aquellas relativas a terminología y las que se refieran a su cumplimiento y aplicación \cite{Normas}

\textbf{Tiny Machine Learning (TML):} Se enfoca en desarrollar modelos de aprendizaje automático que pueden ser ejecutados en microcontroladores y otros dispositivos con restricciones de memoria y energía por su tamaño \cite{IA}

\textbf{HSV}: En el modelo de color HSV, un color se define por su tono (H), su saturación (S) y su luminosidad (V), por lo que se parece más a la percepción del color humano que a los modelos de color aditivos y sustractivos. Es fácil ajustar un color por su saturación y brillo \cite{ModeloColor}.

\textbf{Edge Impulse}: Edge Impulse es una plataforma para desarrollar algoritmos de aprendizaje máquina enfocados a implementarse en sistemas embebidos como microcontroladores o computadoras con recursos reducidos \cite{EdgeImpulse}.

\textbf{ESP32 Cam}: ESP32-CAM, es un dispositivo que puede llamarse un todo en uno. Aparte de la conectividad Wifi y Bluetooth que viene de fábrica, pines GPIO, se le han añadido dos opciones más. Lleva integrada una pequeña cámara de vídeo y una conexión para una tarjeta MicroSD, donde podremos almacenar fotos o vídeos.

Históricamente el limón en México se empezó a cultivar en el siglo XX en el estado de Michoacán \cite{Frutas} y pronto se extendió hacia los estados vecinos. En esos años los limones eran seleccionado por su tamaño y por su grado de madurez, aunque también debían de cumplir la característica de ser mayores de 41mm, aquellos limones los cuales presentaran un color amarillento o con colores desuniformes o manchado se separaban y eran enviados al mercado local, mientras que los limones de color uniforme se empacaban para su exportación nacional e internacional \cite{Frutas}. 

Este proceso era realizado por personas con conocimientos en el campo agrícola, pero a pesar de esto, el error humano persistía y en ocasiones los limones de exportación no cumplían con las características necesarias para su embalaje.

Este proyecto de selección de limones se enfoca en separar los limones en “buen estado” de los limones en “mal estado”, es decir los limones en “buen estado” serán limones para la venta en el mercado local y para la exportación, mientras que los limones en mal estado serán desechados ya que simplemente no son aptos para el consumo humano.

Actualmente los limones que se producen en el país están regulados bajo la NOM-FF-331-A-1981 la cual define características que deben cumplir los limones mexicanos para ser consumidos por el humano o comercializados.

En el 2018 el Ingeniero en sistemas computacionales Juan Orlando Salazar Campos de la Universidad Privada del Norte publicó la tesis “DESARROLLO DE UN SISTEMA DE VISIÓN ARTIFICIAL PARA REALIZAR UNA CLASIFICACIÓN UNIFORME DE LIMONES” en la cual se desarrolla un sistema utilizando la visión artificial para clasificar los limones de manera uniforme

“Las formas y dimensiones de los limones a ser analizados están sujetos al códex de la lima-limón de la Organización de Comida y Agricultura de las Naciones Unidas” \cite{Vision}.

Afirma que: “en ese año no existía tecnología de información asociada al proceso de clasificación de limones, esto brinda la posibilidad de explorar alternativas basadas en áreas de la computación que ayude en el proceso de clasificación con visión artificial” \cite{Vision}.

El algoritmo desarrollado en este proyecto obtuvo resultados bastante satisfactorios con una eficacia del 83.9\%, sensibilidad de 82:8\% y especificidad del 100\%. Comprobó que su hipótesis planteada, la cual sostenía que un sistema de visión artificial permite una clasificación uniforme de limones \cite{Vision}.

El entorno fotográfico que utilizó para la toma de muestras constaba de un trípode con una lámpara LED y un fondo milimetrado donde las imágenes tomadas eran enviadas a la PC para ser procesadas con HSV, realzando el contraste y reduciendo el ruido, posteriormente aplicando filtros para segmentarla por el umbral, descripción de la región y el reconocimiento e interpretación de la decisión teórica.
Se realizaron pruebas con 385 imágenes en las cuales contenían limones en diferentes estados de madurez, en un ambiente controlado “bajo criterio personal”

Los resultados de este proyecto fueron los siguientes: Verdadero positivo(VP) de 298 limones con dimensiones y colores que cumplen la norma, Verdadero negativo(VN) de 25 limones con dimensiones y color que no cumple la norma y por último Falso positivo (FP) de 0 limones con otras características y Falso Negativo(FN) con 62 limones los cuales no se les asignó una característica correcta.

Con los resultados obtenidos es como concluye el proyecto del clasificador de limones, si bien esto fue un buen proyecto, solo se presentó el algoritmo que clasificaba los limones, pero aún quedaba en duda cómo aplicar este algoritmo a una línea de producción, es decir, como fabricar una cinta transportadora en la cual en una parte se efectuará la clasificación mediante machine learning.

Un año antes del publicación de la tesis anterior, se presento un documento en el cual se presentaba otro proyecto idéntico el cual se llamo“ SISTEMA AUTOMATICO DE SELECCIÓN DE LIMÓN (Citrus  Latifolia Tanaka) BASADO EN DISCRIMINACIÓN POR COLOR. Este sistema está basado en el procesamiento digital de imágenes para seleccionar limones persa el cual en México es el de mayor producción. Se basa en una camara de inspección que establece las condiciones necesarias para el procesamiento de iamgenes realizando el analisis y segmentacion de color, asi como tambien analisis morfologico del fruto, el cual dtermina sus caractersiticas y calidad siendo un sistema de bajo costo debido a que soporta su operacion en una computadora personal y una camara web de alta definición \cite{SistemaSeleccion}.

Su sistema consta de un depósito de frutos, una cámara de inspección y un depósito de frutos clasificados, los limones se mueven mediante una banda transportadora. En el depósito de frutos, se ingresan una variedad de limones comprados en mercados de Orizaba, Veracruz, los limones ya vienen lavados y sin imperfecciones producidos por el ambiente de transporte (maleza, tierra o manchas de otros frutos), estos limones caen en una banda transportadora construida con rodillos con un sistema que hace que los limones vayan uno detrás del otro, la banda transportadora pasa por una cámara de inspección en la cual se inspeccionan y se lleva a cabo el proceso de selección, posteriormente, al final de la banda una válvula electrónica separa los limones de acuerdo a su tonalidad.

Para un mejor procesado de las imágenes mediante HSV se tiene un control de iluminación, las imágenes son segmentadas por color, posteriormente se separan por regiones de interés, se detectan los contornos y por último los círculos. De esta manera es como se van clasificando los limones por su tonalidad de color.

Como resultado y conclusión de este proyecto: Se presenta una alternativa económica basada en el procesamiento de imágenes en tiempo real que utiliza únicamente una computadora personal, una cámara web y una caja cerrada con iluminación controlada como cámara de inspección este dispositivo fue desarrollado con la intención de dotar a los productores en pequeño, de un equipo que les ayude en su tarea de selección con el fin de promover la exportación e incrementar sus beneficios \cite{SistemaSeleccion}.

Con la consulta del trabajo anterior y tesis, nos podemos dar una idea de como empezar este proyecto de selector de limones, a pesar de tener un mismo objetivo que es clasificar los limones en buenos y malos, este proyecto se diferencia en que se utilizara “Tiny Machine Learning, es decir mediante ML se clasificaron los limones y todo estará embebido en un sistema independiente sin necesidad de que sea intervenido o ocupe opinión de un humano para su proceso de selección, no tendrá la necesidad de tener conexión a una computadora. Se utilizará el modelo HSV para la clasificación de los limones con ayuda de la plataforma de Edge Impulse y con una ESP32 Cam para la obtención de muestras y posterior para la clasificación. Utilizando materiales como servomotores, bandas transportadoras y recolectores se hará el sistema para el transporte del limón hacia el área de clasificación. 


\clearpage
\section{Descripción del problema}
Está problemática se presenta porque el identificar cada uno de los limones de un cultivo sería un proceso demasiado largo y costoso. Es por esto que es importante hacer este proceso un poco más autónomo y sin temor al error humano ya que no podemos tener la certeza de cuándo está lista para almacenarse. Esto suele ser difícil ya que si se almacena alguno que no está en condiciones óptimas puede afectar al resto del cultivo afectando así la producción completa ocasionando grandes pérdidas económicas y quitando credibilidad al proveedor de vender fruta de calidad. 

\clearpage
\section{Motivación}
La automatización de los procesos productivos en diferentes áreas es un reto que se está implementando para hacerlos más eficientes y reducir los errores en el trabajo manual. Este interés o necesidad ha llegado a la industria agrícola y en este proyecto se plantea eficientar la selección de frutas, específicamente el limón, cuya maduración es difícil de determinar de manera precisa de forma manual lo que aumenta la posibilidad de errores y que se genere desperdicio alimenticio debido a las decisiones inexactas ya que el limón es parte de la comida de la mayoría de las personas. 

El desarrollo de una máquina capaz de automatizar este proceso no solo cumpliria la tarea de eficientar la selección de limones, sino también una oportunidad para reducir pérdidas económicas y ofrecer productos de mayor calidad ya que si se consume un limón que aun no esta listo o que está pasado de tiempo esto puede dejar un mal sabor de boca o simplemente arruinarnos nuestra comida o enfermarnos.


\clearpage
\section{Justificación}

La idea de desarrollar un selector de limones es que se clasifiquen los limones de acuerdo a su color para determinar si ha llegado al punto correcto de madurez y asegurarse de que la calidad de los frutos sea la óptima. La implementación de un sistema automatizado para la selección de limones reduciría el error humano y lograría un proceso eficiente. 
Además, no solo se trata de agilizar el trabajo, sino también de mejorar la manera en que se manejan los limones para tener un mejor control sobre el proceso y a garantizar que los limones seleccionados estén en las mejores condiciones posibles para el siguiente paso, ya sea almacenamiento o procesamiento.


\clearpage
\section{Alcances y limitaciones}

\textbf{Alcances}
\begin{itemize}
    \item Prototipo funcional: Crear una versión operativa del sistema que automatice el proceso de selección de limones. 
    \item Movimiento automatizado: Manejo automático de las piezas del sistema para transportar y procesar limones. 
    \item Madurez basada en color: Detección por medio del análisis de colores para determinar el estado de madurez. 
    \item Control y monitoreo en tiempo real: Seguimiento del sistema en funcionamiento, mostrando el estado del proceso.
    \item Almacenamiento y transmisión de datos: Guardar y enviar información del estado de madurez en los limones.
    \item Uso de tecnología de bajo costo: Utilización de componentes accesibles económicamente o componentes que se encuentran en nuestra posesión. 
    \item Conectividad inalámbrica: Permitir que el sistema se comunique de forma remota sin necesidad de cables.
    \item Alertas visuales o sonoras: Añadir señales que indiquen el estado del proceso o la detección de la madurez de los limones.
    \item Reconocimiento de patrones de maduración: Aplicación de técnicas de procesamiento de imágenes para determinar colores que indiquen un grado de madurez.
\end{itemize}

\textbf{Limitaciones}
\begin{itemize}
    \item Limitación de tiempo: Restricción en la disponibilidad de tiempo para completar el proyecto dentro de un plazo determinado. 
    \item Escalabilidad limitada: El sistema será aplicado para trabajar con un tipo de limón.
    \item Dependencia del entorno: La funcionalidad puede verse afectada por factores externos, como la iluminación. 
    \item Procesamiento de imagen en la ESP32-CAM: La cámara del ESP32 no contiene demasiada potencia para procesar imágenes complejas. 
    \item Sincronización de motores y cámara: Problemas de sincronización y ajuste de los motores con la cámara. 
    \item Detección de imperfecciones internas: Que el sistema detecte la cáscara del limón y no identifique los posibles problemas internos de la fruta. 
    \item Rango de detección de la ESP32-CAM : El rango de alcance de la cámara no es muy extenso, lo que puede afectar su precisión.
    \item Entrenamiento del modelo: Errores en el entrenamiento del modelo por datos insuficientes o ruido. 
    \item Limitada capacidad de expansión: Por el momento no está pensado para añadir funciones nuevas.
    \item Calidad de los limones: Variaciones de los limones de prueba para la creación del dataset.
\end{itemize}


%-----------------------------------------------------------------------------------------------------------------
% REFERENCIAS


\clearpage
%Let's cite! The Einstein's journal paper \cite{dirac} and the Dirac's 
%book \cite{einstein} are physics related items. 

%\Urlmuskip=0mu plus 1mu\relax
\addcontentsline{toc}{section}{Referencias} 
\printbibliography
 
\end{document}
