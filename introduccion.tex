
\section{Introducción}
La industria agrícola tiene un proceso delicado que empieza desde la selección hasta la cosecha y almacenamiento. En particular, los limones forman parte de la dieta de muchas personas y sus características logran darle un toque refrescante a comidas y bebidas. Tomar en cuenta la vida útil del limón y la necesidad de ofrecer productos de calidad permite garantizar la reducción del desperdicio y la satisfacción de los consumidores. \\\\
El problema radica en la dificultad de detectar cuándo un limón ha alcanzado el punto óptimo de maduración; en ocasiones este proceso se realiza manualmente y puede generar imprecisión en el proceso además de no ser lo suficientemente rápido. \\\\
En la actualidad, la automatización se ha convertido en una solución para optimizar numerosos procesos, reduciendo así la participación humana y los errores asociados. Una de las áreas que se ha visto involucrada en estos cambios tecnológicos ha sido la agricultura, en la que se ha hecho uso de drones para monitorear el crecimiento de los cultivos, detectar problemas en el campo o realizar tareas programadas; robots para realizar tareas como la cosecha de frutas y verduras o trabajar en terrenos difíciles o peligrosos; sensores y sistemas de control que detecten la humedad del suelo, temperatura y demás factores ambientales para ayudar a los agricultores a ajustar la cantidad de agua y nutrientes que se aplican a los cultivos. \\\\
Con este nivel de tecnología, es que se pueden desarrollar soluciones adecuadas utilizando dispositivos para el procesamiento de imágenes. Entre ellos tenemos la idea de crear una banda transportadora automatizada, en la que se vayan colocando los limones, y una cámara ESP32-CAM, que permita la visualización para la revisión de los limones. Este mecanismo requiere la combinación del hardware y software para reconocimiento de colores que indiquen la madurez del limón, tomando en cuenta nuestro principal objetivo que es mejorar el proceso en la selección de limones disminuyendo el error humano. 
